{%
  \caption{Convenção adaptada para expressar BRDFs}%
  \label{tabela:simbolos}
}{%
    \begin{tabular}{ c c }
   
	\textbf{Símbolo} 								& \textbf{Significado}\\
    
	$\vec{\omega_{i}}$ 								& direção da luz incidente \\
    
	$\vec{\omega_{o}}$ 								& direção da luz refletida\\
  
    $\vec{\omega_{i}}$ = ($\theta_{i}, \phi_{i})$	& coordenadas polares de $\vec{\omega_{i}}$\\
    
	$\vec{\omega_{o}}$ = ($\theta_{o}, \phi_{o})$	& coordenadas polares de $\vec{\omega_{o}}$\\
    
    $\theta_{i}$									& ângulo de elevação da direção da luz incidente\\
    
    $\theta_{o}$									& ângulo de elevação da direção da luz refletida\\
    
    $\phi_{i}$          							& ângulo azimutal da direção da luz incidente\\
   
    $\phi_{o}$         								& ângulo azimutal  da direção da luz refletida\\
	
	$f$				    							& BRDF aproximada\\
	
	$\vec{n}$										& vetor normal da superfície\\
	
	$\vec{h}$										& vetor halfway\\
   
	$\theta_{h}$									& ângulo entre $\vec{n}$ e $\vec{h}$\\
   
	$\theta_{d}$									& ângulo entre $\vec{\omega_{i}}$ ou $\vec{\omega_{o}}$ e $\vec{h}$\\
	
	$\rho_{d}$										& coeficiente de difusão em RGB\\
	
	$\rho_{s}$										& coeficiente especular em RGB\\
    
    \midrule
    
    \end{tabular}%
}
\end{table}